\documentclass{article}
\usepackage[english]{babel}
\usepackage[letterpaper,top=2cm,bottom=2cm,left=3cm,right=3cm,marginparwidth=1.75cm]{geometry}
\usepackage[T2A]{fontenc}
\usepackage[utf8]{inputenc}
\usepackage[russian]{babel}
\usepackage{amsmath}
\usepackage{graphicx}
\usepackage[colorlinks=true, allcolors=blue]{hyperref}
\date{}
\title{Tasks}
\author{Codeforces}
\begin{document}
\maketitle
\section{\href{https://codeforces.com/problemset/problem/753/A}{A. Santa Claus and Candies}}

\begin{itemize}
\item \textbf{time limit per test:} 1 second
\item \textbf{memory limit per test:} 256 megabytes
\end{itemize}
Santa Claus has $n$ candies, he dreams to give them as gifts to children. What is the maximal number of children for whose he can give candies if Santa Claus want each kid should get distinct positive integer number of candies. Santa Class wants to give all $n$ candies he has.

\subsection*{Input}
 The only line contains positive integer number $n$ ( $1 \leq n \leq 1000$ ) — number of candies Santa Claus has.

\subsection*{Output}
 Print to the first line integer number $k$ — maximal number of kids which can get candies. Print to the second line $k$ distinct integer numbers: number of candies for each of $k$ kid. The sum of $k$ printed numbers should be exactly $n$ . If there are many solutions, print any of them.

\subsection*{Sample 1}
\begin{itemize}
\item \textbf{Sample input:} 
5
\item \textbf{Sample output:} 
2
2 3
\end{itemize}
\subsection*{Sample 2}
\begin{itemize}
\item \textbf{Sample input:} 
9
\item \textbf{Sample output:} 
3
3 5 1
\end{itemize}
\subsection*{Sample 3}
\begin{itemize}
\item \textbf{Sample input:} 
2
\item \textbf{Sample output:} 
1
2
\end{itemize}
\section{\href{https://codeforces.com/problemset/problem/1616/D}{D. Keep the Average High}}

\begin{itemize}
\item \textbf{time limit per test:} 1.5 seconds
\item \textbf{memory limit per test:} 256 megabytes
\end{itemize}
You are given an array of integers $a_1, a_2, \ldots, a_n$ and an integer $x$. You need to select the maximum number of elements in the array, such that for every subsegment $a_l, a_{l + 1}, \ldots, a_r$ containing strictly more than one element $(l < r)$, either: At least one element on this subsegment is not selected, or $a_l + a_{l+1} + \ldots + a_r \geq x \cdot (r - l + 1)$.

\subsection*{Input}
 The first line of input contains one integer $t$ ($1 \leq t \leq 10$): the number of test cases. The descriptions of $t$ test cases follow, three lines per test case. In the first line you are given one integer $n$ ($1 \leq n \leq 50\,000$): the number of integers in the array. The second line contains $n$ integers $a_1, a_2, \ldots, a_n$ ($-100\,000 \leq a_i \leq 100\,000$). The third line contains one integer $x$ ($-100\,000 \leq x \leq 100\,000$).

\subsection*{Output}
 For each test case, print one integer: the maximum number of elements that you can select.

\subsection*{Sample 1}
\begin{itemize}
\item \textbf{Sample input:} 
4
5
1 2 3 4 5
2
10
2 4 2 4 2 4 2 4 2 4
3
3
-10 -5 -10
-8
3
9 9 -3
5
\item \textbf{Sample output:} 
4
8
2
2
\end{itemize}
\section{\href{https://codeforces.com/problemset/problem/2023/D}{D. Many Games}}

\begin{itemize}
\item \textbf{time limit per test:} 2 seconds
\item \textbf{memory limit per test:} 256 megabytes
\end{itemize}
Recently, you received a rare ticket to the only casino in the world where you can actually earn something, and you want to take full advantage of this opportunity. The conditions in this casino are as follows: There are a total of $n$ games in the casino. You can play each game at most once . Each game is characterized by two parameters: $p_i$ ($1 \le p_i \le 100$) and $w_i$ — the probability of winning the game in percentage and the winnings for a win. If you lose in any game you decide to play, you will receive nothing at all (even for the games you won). You need to choose a set of games in advance that you will play in such a way as to maximize the expected value of your winnings. In this case, if you choose to play the games with indices $i_1 < i_2 < \ldots < i_k$, you will win in all of them with a probability of $\prod\limits_{j=1}^k \frac{p_{i_j}}{100}$, and in that case, your winnings will be equal to $\sum\limits_{j=1}^k w_{i_j}$. That is, the expected value of your winnings will be $\left(\prod\limits_{j=1}^k \frac{p_{i_j}}{100}\right) \cdot \left(\sum\limits_{j=1}^k w_{i_j}\right)$. To avoid going bankrupt, the casino owners have limited the expected value of winnings for each individual game. Thus, for all $i$ ($1 \le i \le n$), it holds that $w_i \cdot p_i \le 2 \cdot 10^5$. Your task is to find the maximum expected value of winnings that can be obtained by choosing some set of games in the casino.

\subsection*{Input}
 The first line contains a single integer $n$ ($1 \le n \le 2 \cdot 10^5$) — the number of games offered to play. The $i$-th of the following $n$ lines contains two integers $p_i$ and $w_i$ ($1 \leq p_i \leq 100$, $1 \leq w_i, p_i \cdot w_i \leq 2 \cdot 10^5$) — the probability of winning and the size of the winnings in the $i$-th game.

\subsection*{Output}
 Output a single number — the maximum expected value of winnings in the casino that can be obtained by choosing some subset of games. Your answer will be accepted if the relative or absolute error does not exceed $10^{-6}$. Formally, if $a$ is your answer and $b$ is the jury's answer, it will be accepted if $\frac{|a-b|}{\max(b, 1)} \le 10^{-6}$.

\subsection*{Sample 1}
\begin{itemize}
\item \textbf{Sample input:} 
3
80 80
70 100
50 200
\item \textbf{Sample output:} 
112.00000000
\end{itemize}
\subsection*{Sample 2}
\begin{itemize}
\item \textbf{Sample input:} 
2
100 1
100 1
\item \textbf{Sample output:} 
2.00000000
\end{itemize}
\subsection*{Sample 3}
\begin{itemize}
\item \textbf{Sample input:} 
4
1 100
2 1000
2 100
3 1
\item \textbf{Sample output:} 
20.00000000
\end{itemize}
\subsection*{Sample 4}
\begin{itemize}
\item \textbf{Sample input:} 
5
34 804
78 209
99 191
61 439
90 79
\item \textbf{Sample output:} 
395.20423800
\end{itemize}
\section{\href{https://codeforces.com/problemset/problem/1692/H}{H. Gambling}}

\begin{itemize}
\item \textbf{time limit per test:} 2 seconds
\item \textbf{memory limit per test:} 256 megabytes
\end{itemize}
Marian is at a casino. The game at the casino works like this. Before each round, the player selects a number between $1$ and $10^9$. After that, a dice with $10^9$ faces is rolled so that a random number between $1$ and $10^9$ appears. If the player guesses the number correctly their total money is doubled, else their total money is halved. Marian predicted the future and knows all the numbers $x_1, x_2, \dots, x_n$ that the dice will show in the next $n$ rounds. He will pick three integers $a$, $l$ and $r$ ($l \leq r$). He will play $r-l+1$ rounds (rounds between $l$ and $r$ inclusive). In each of these rounds, he will guess the same number $a$. At the start (before the round $l$) he has $1$ dollar. Marian asks you to determine the integers $a$, $l$ and $r$ ($1 \leq a \leq 10^9$, $1 \leq l \leq r \leq n$) such that he makes the most money at the end. Note that during halving and multiplying there is no rounding and there are no precision errors. So, for example during a game, Marian could have money equal to $\dfrac{1}{1024}$, $\dfrac{1}{128}$, $\dfrac{1}{2}$, $1$, $2$, $4$, etc. (any value of $2^t$, where $t$ is an integer of any sign).

\subsection*{Input}
 The first line contains a single integer $t$ ($1 \leq t \leq 100$) — the number of test cases. The first line of each test case contains a single integer $n$ ($1 \leq n \leq 2\cdot 10^5$) — the number of rounds. The second line of each test case contains $n$ integers $x_1, x_2, \dots, x_n$ ($1 \leq x_i \leq 10^9$), where $x_i$ is the number that will fall on the dice in the $i$-th round. It is guaranteed that the sum of $n$ over all test cases does not exceed $2\cdot10^5$.

\subsection*{Output}
 For each test case, output three integers $a$, $l$, and $r$ such that Marian makes the most amount of money gambling with his strategy. If there are multiple answers, you may output any of them.

\subsection*{Sample 1}
\begin{itemize}
\item \textbf{Sample input:} 
4
5
4 4 3 4 4
5
11 1 11 1 11
1
1000000000
10
8 8 8 9 9 6 6 9 6 6
\item \textbf{Sample output:} 
4 1 5
1 2 2
1000000000 1 1
6 6 10
\end{itemize}
\section{\href{https://codeforces.com/problemset/problem/2115/D}{D. Gellyfish and Forget-Me-Not}}

\begin{itemize}
\item \textbf{time limit per test:} 2 seconds
\item \textbf{memory limit per test:} 1024 megabytes
\end{itemize}
Gellyfish and Flower are playing a game. The game consists of two arrays of $n$ integers $a_1,a_2,\ldots,a_n$ and $b_1,b_2,\ldots,b_n$, along with a binary string $c_1c_2\ldots c_n$ of length $n$. There is also an integer $x$ which is initialized to $0$. The game consists of $n$ rounds. For $i = 1,2,\ldots,n$, the round proceeds as follows: If $c_i = \mathtt{0}$, Gellyfish will be the active player. Otherwise, if $c_i = \mathtt{1}$, Flower will be the active player. The active player will perform exactly one of the following operations: Set $x:=x \oplus a_i$. Set $x:=x \oplus b_i$. Here, $\oplus$ denotes the bitwise XOR operation . Gellyfish wants to minimize the final value of $ x $ after $ n $ rounds, while Flower wants to maximize it. Find the final value of $ x $ after all $ n $ rounds if both players play optimally.

\subsection*{Input}
 Each test contains multiple test cases. The first line contains the number of test cases $t$ ($1 \le t \le 10^4$). The description of the test cases follows. The first line of each test case contains a single integer $n$ ($1 \leq n \leq 10^5$) — the number of rounds of the game. The second line of each test case contains $n$ integers $a_1, a_2, \ldots, a_n$ ($0 \leq a_i < 2^{60}$). The third line of each test case contains $n$ integers $b_1, b_2, \ldots, b_n$ ($0 \leq b_i < 2^{60}$). The fourth line of each test case contains a binary string $c$ of length $n$. It is guaranteed that the sum of $n$ over all test cases does not exceed $10^5$.

\subsection*{Output}
 For each test case, output a single integer — the final value of $ x $ after all $ n $ rounds.

\subsection*{Sample 1}
\begin{itemize}
\item \textbf{Sample input:} 
5
1
0
2
0
2
12 2
13 3
11
3
6 1 2
6 2 3
010
4
1 12 7 2
4 14 4 2
0111
9
0 5 10 6 6 2 6 2 11
7 3 15 3 6 7 6 7 8
110010010
\item \textbf{Sample output:} 
0
15
6
11
5
\end{itemize}
\end{document}
